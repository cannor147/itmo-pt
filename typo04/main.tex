\documentclass[a4paper,fontsize=13pt]{article}

\usepackage[utf8]{inputenc} % для кодировки
\usepackage[russian]{babel} % для того, чтобы писать русский текст
\usepackage{amsmath} % для команды equation*
\usepackage{hyperref} % для вставки ссылок
\usepackage{graphicx}
\usepackage[left=2cm,right=2cm,top=2cm,bottom=2cm,bindingoffset=0cm]{geometry}
\usepackage{caption}
\usepackage{tabto}
\usepackage{scrextend}
\usepackage{multicol} % список из нескольких колонок
\usepackage{amsfonts} % множества чисел
\usepackage{fancyvrb}

\title{Метод Монте-Карло}
\date{16 июня 2020 г.}
\author{Ерофей Башунов, M3336 (вариант 4)}

\begin{document}
\pagenumbering{gobble} % отключаем нумерацию страниц

\maketitle % титульная страница
\newpage % страница с решением первой задачи
\pagenumbering{arabic} % включаем нумерацию страниц

\section*{Задание №1}

\subsection*{Условие}

\tab Методом Монте-Карло оценить объем части тела $\{F(\overline{x}) \le c \}$, заключѐнной в $k$-мерном кубе с ребром $[0, 1]$. Функция имеет вид $F(\overline{x}) = f(x_1) + f(x_2) + \cdots + f(x_k)$. Для выбранной надежности $\gamma \ge 0.95$ указать асимптотическую точность оценивания и построить асимптотический доверительный интервал для истинного значения объёма.

Используя объём выборки $n=10^4$ и $n=10^6$, оценить скорость сходимости и показать, что доверительные интервалы пересекаются.

$$f(x) = \ln(ax + 1) \quad\quad\quad k = 3 \quad\quad\quad c = 2.8 \quad\quad\quad a = 3$$

\subsection*{Решение}

\tab Следующий код на языке Octave является решением поставленной задачи.

\VerbatimInput{cube.m}

\begin{center}
\begin{tabular}{| c | c | c |}
\hline
 $n$ & $10^4$ & $10^6$ \\ 
 \hline
 $p$ & $0.637100$ & $0.628023$ \\  
 \hline
 $d$ & $0.009424$ & $0.000947$ \\  
 \hline
 $I$ & $[0.627676, 0.646524]$ & $[0.627076, 0.628970]$ \\
 \hline
 $length(I)$ & $0.018848$ & $0.001894$ \\
 \hline
\end{tabular}
\end{center}

При увеличении выборки в $100$ раз доверительный интервал уменьшился приблизительно в $10$ раз. Доверительные интервалы для выборок пересекаются на промежутке $[0.627676, 0.628970]$. 

\section*{Задание №2}

\subsection*{Условие}

Аналогично построить оценку интегралов (представить интеграл как математическое ожидание функции, зависящей от случайной величины с известной плотностью) и для выбранной надежности указать асимптотическую точность оценивания и построить асимптотический доверительный интервал для истинного значения интеграла.

$$\text{а.} \int_{1}^{4} 3^{-x^2} dx \quad\quad\quad \text{b.} \int_{0}^{+\infty} x^{\frac{2}{3}} e^{-2x} dx$$

\subsection*{Решение}

\subsubsection*{Пункт a.}

\VerbatimInput{int_a.m}

\begin{center}
\begin{tabular}{| c | c | c |}
\hline
 $n$ & $10^4$ & $10^6$ \\ 
 \hline
 $p$ & $0.115727$ & $0.116991$ \\  
 \hline
 $d$ & $0.004390$ & $0.000443$ \\  
 \hline
 $I$ & $[0.111337, 0.120117]$ & $[0.116548, 0.117434]$ \\
 \hline
 $length(I)$ & $0.00878$ & $0.000886$ \\
 \hline
\end{tabular}
\end{center}

При увеличении выборки в $100$ раз доверительный интервал уменьшился приблизительно в $10$ раз. Доверительные интервалы для выборок пересекаются на промежутке $[0.116548, 0.117434]$. Реальное значение интеграла равно $0.116901$ и лежит в пределах обоих доверительных интервалов.

\subsubsection*{Пункт b.}

\VerbatimInput{int_b.m}

\begin{center}
\begin{tabular}{| c | c | c |}
\hline
 $n$ & $10^4$ & $10^6$ \\ 
 \hline
 $p$ & $0.284764$ & $0.284422$ \\  
 \hline
 $d$ & $0.003797$ & $0.000379$ \\  
 \hline
 $I$ & $[0.280967, 0.288561]$ & $[0.284043, 0.284801]$ \\
 \hline
 $length(I)$ & $0.007594$ & $0.000758$ \\
 \hline
\end{tabular}
\end{center}

При увеличении выборки в $100$ раз доверительный интервал уменьшился приблизительно в $10$ раз. Доверительные интервалы для выборок пересекаются на промежутке $[0.284043, 0.284801]$. Реальное значение интеграла равно $0.284347$ и лежит в пределах обоих доверительных интервалов.

\end{document}
